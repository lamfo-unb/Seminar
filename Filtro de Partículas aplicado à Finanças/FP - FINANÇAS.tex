\documentclass{beamer}
% Setup appearance:
\usepackage{lmodern}
\usepackage[labelformat=empty,font=scriptsize,skip=0pt,justification=justified,singlelinecheck=false]{caption}
\setbeamertemplate{footline}[frame number]

 
%encoding
%--------------------------------------
\usepackage[utf8]{inputenc}
\usepackage[T1]{fontenc}
%--------------------------------------
 
%Portuguese-specific commands
%--------------------------------------
\usepackage[portuguese]{babel}
%--------------------------------------
 
%Hyphenation rules
%--------------------------------------
\usepackage{hyphenat}
\hyphenation{mate-mática recu-perar}


%remove the icon
\setbeamertemplate{bibliography item}{}

%remove line breaks
\setbeamertemplate{bibliography entry title}{}
\setbeamertemplate{bibliography entry location}{}
\setbeamertemplate{bibliography entry note}{}



\title[Artigo]{Filtro de Partículas em Finanças}
\author[Igor Nascimento]{Igor Nascimento}
\institute[LAMFO]{Laboratório de Aprendizado de Máquina em Finanças e Organizações}
\date[2018]{05/03/2018}

\begin{document}

\begin{frame}
  \titlepage
\end{frame}

\section{Introdução}

\begin{frame}{Visão geral}


\vspace{.15cm}
\begin{enumerate}
\item Modelos de Espaços de Estados
\vspace{.15cm}
\item Filtro de Partículas
\vspace{.15cm}
\item Finanças
\end{enumerate}

\end{frame}

\begin{frame}{Série Temporal}


\begin{enumerate}
\item transversal
\item longitudinal
\end{enumerate}

\end{frame}

\begin{frame}{Objetivo}

\begin{itemize}
\item Apresenta os desenvolvimentos recentes da ferramenta Filtro de Partículas e principais aplicações.
\end{itemize}

\end{frame}

\begin{frame}{Expectativa}

\begin{itemize}
\item Entender o método
\item Relacionar à aplicações em Finanças
\item Conhecer as principais referências
\end{itemize}

\end{frame}




\section{Modelos de Espaços de Estados}

\begin{frame}{Modelos de Espaços de Estado}

O que são modelos de espaços de estados

\end{frame}


\begin{frame}{Modelos de Espaços de Estado}

Por que isso é interessante ?

\end{frame}


\section{Filtro de Partículas}

\subsection{Bootstrap}

\begin{frame}{Bootstrap}

O que é Bootstrap

\end{frame}




\begin{frame}{Sistemas Complexos}


Por que isso é interessante ?

\end{frame}



\subsection{Filtro de Partículas}

\begin{frame}{Filtro de Partículas}

O que é Filtro de Partículas

\end{frame}



\begin{frame}{Série Temporal}


\begin{enumerate}

\item Complexidade (transversal)

\item  Complexidade (longitudinal)

\end{enumerate}

\end{frame}


\subsection{Implmentações}

\begin{frame}{Implmentação}

\begin{itemize}

\item R

\item Python


\end{itemize}

\end{frame}



\section{Aplicação}


\begin{frame}{Aplicações}

\begin{itemize}

\item Retorno

\item Volatilidade
\end{itemize}

\end{frame}


\begin{frame}{Retorno}

Modelo para retorno
\end{frame}


\begin{frame}{Volatilidade}
Modelo para volatilidade
\end{frame}

\section{Considerações finais}
\begin{frame}{Considerações finais}

Flexibiliza

\end{frame}



\begin{frame}[allowframebreaks]%in case more than 1 slide needed
    {\tiny
    \bibliographystyle{apalike}
    \bibliography{bibli}
    }
\end{frame}

\end{document}



